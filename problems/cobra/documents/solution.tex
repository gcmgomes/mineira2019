% Para adicionar uma imagem ou incluir um arquivo .tex você precisa 
% adicionar \CWD
% caminho relativo (ao documento principal) do diretório.
% 
% Exemplo:
% \begin{figure}
%   \includegraphics{\CWD/imagens/exemplo.pdf}
% \end{figure}

No problema "Cobra de Dominós", dado um dominó com peças que tenham $N$ valores distintos (de $0$ até $N-1$, por exemplo) a equipe deve desenvolver o raciocínio matemático para reconhecer a construção da sequência que descreve o número máximo de peças, $P(N)$, usadas para se construir uma cobra usando as regras de encaixe (mesmo valor dos lados das peças).

Por exemplo:

* Para $N = 1, P(N) = 1$.  Uma maior cobra possível de se contruir é |0-0|
* Para $N = 2, P(N) = 3$.  Uma maior cobra possível de se contruir é |0-0||0-1||1-1|
* Para $N = 3, P(N) = 6$.  Uma maior cobra possível de se contruir é |0-0||0-1||1-1||1-2||2-2||2-0|
* Para $N = 4, P(N) = 9$.  Uma maior cobra possível de se contruir é |0-0||0-1||1-1||1-2||2-2||2-2||2-3||3-3||3-0| (não usa |1-3|)
* Para $N = 5, P(N) = 15$. Uma maior cobra possível de se contruir é |0-0||0-1||1-1||1-2||2-2||2-2||2-3||3-3||3-0||0-4||4-1||1-3||3-4||4-4||4-2|
* Para $N = 6, P(N) = 19$. Uma maior cobra possível de se contruir é |5-0||0-0||0-1||1-1||1-2||2-2||2-2||2-3||3-3||3-0||0-4||4-1||1-3||3-4||4-4||4-2||2-5||5-5||5-1| (não usa |3-5|)

Usando a mesma técnica construtiva, percebe-se que $P(7) = 28, P(8) = 33, P(9) = 45, \ldots$

Essa sequência pode ser descrita por

$P(N) = C(N, 2) + N$, (se N é ímpar) ou $P(N) = C(N, 2) + N/2$, (se N é par);

que pode ser simplificada para

$P(N) = (( (-1)^N ) * (2 - N) + 2 + N + 2*N^2)/4$.
