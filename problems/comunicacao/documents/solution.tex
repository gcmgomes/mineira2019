%Vamos resolver primeiro o caso em que nenhuma ligação é dada previamente. Se numerarmos os pinos de A e B de 1 a N, a paridade do número de cruzamentos de uma conexão que liga o pino P em A ao pino Q em B é mesma paridade de P + Q. Então, teremos uma configuração válida sse ligarmos os pares nos pares e os ímpares nos ímpares. Isso nos deixa com dois problemas separados. Pra cada um deles, precisamos contar o número de permutações de um certo tamanho. Ou seja, nesse caso a resposta seria X! * (N - X)!, onde X é o número de números pares de 1 a N.

%Agora vamos considerar as conexões prévias. Depois de verificar se não tem nenhum pino com dois fios ou mais ou se alguma ligação ligou um pino par a um pino ímpar, basta contar o número de números pares restantes e o número de número ímpares restantes e realizar um cálculo similar ao anterior.
