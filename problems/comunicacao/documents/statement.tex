Luci visitou recentemente o Museu da História da Computação, onde teve a oportunidade de apreciar várias exposições. Em uma delas, conheceu várias calculadores antigas como a \emph{Step Reckoner}, criada por volta de 1672. Em outra, viu supercomputadores do passado, como os famosos Cray. Também conheceu várias maneiras de programar computadores antigos, como cartões perfurados e painéis de tomadas.


Um painel de tomadas em particular chamou a atenção de Luci. Nele, há dois blocos com $N$ tomadas cada. Em cada bloco, as tomadas são numeradas de $1$ a $N$, da esquerda para a direita. Usando uma tecnologia peculiar de cabos, o operador conectava as tomadas de um dos blocos às tomadas do outro bloco. Veja o exemplo com $4$ tomadas em cada bloco na figura abaixo.
\begin{figure}[h]
\centering
\includegraphics[scale=0.9]{\CWD/images/fig1.pdf}
\end{figure}

Em cada ciclo do computador ao qual esse painel pertence, cada tomada tem  um sinal elétrico que representa um bit. O propósito desse painel é trocar a ordem dos bits. Por isso, cada tomada de cada bloco deve ser conectada a exatamente uma tomada do outro bloco. Cada conexão deve ser feita com um segmento de cabo ligando as duas tomadas em linha reta. Por restrições físicas relacionadas à tecnologia peculiar dos cabos, a interferência existente sempre que um cabo cruza com outro faz com que o bit transmitido nesse cabo mude de valor. No exemplo da figura acima, o cabo conectando a tomada $1$ do bloco superior à tomada $3$ do bloco inferior intercepta dois cabos. Dessa forma, o bit transmitido por ele será invertido duas vezes, fazendo com que o bit na tomada $1$ do bloco superior seja o mesmo que o na tomada $3$ do bloco inferior.

Como o objetivo do painel é trocar a ordem dos bits, maneiras de conectar as tomadas que fazem com que o bits em duas tomadas conectadas sejam diferentes não são válidas. Ao perceber isso, Luci começou a tentar descobrir de quantas formas era possível conectar as tomadas de maneira válida. Ao perceber que o museu tinha vários painéis desses e que, em alguns deles, algumas conexões já haviam sido feitas, ela achou melhor aproveitar o restante da visita ao museu e deixou a pergunta para você: de quantas formas válidas é possível conectar as tomadas, preservando as conexões já existentes?
Lembrando que uma forma é válida se cada tomada estiver conectada a exatamente uma tomada do bloco oposto e se o bit em tomadas conectadas for o mesmo.

\section*{Observações}
\begin{itemize}
\item Se mais de dois cabos se cruzarem em um mesmo ponto, todos os cruzamentos par a par devem ser considerados.
\end{itemize}
\section*{Entrada}

A primeira linha da entrada contém dois inteiros $N$ e $M$, representando o número de tomadas em cada bloco e o número de conexões já existentes respectivamente.
Seguem $M$ linhas, cada uma contendo dois inteiros $a_i$ e $b_i$, indicando que a tomada $a_i$ do bloco superior está conectada à tomada $b_i$ do bloco inferior.

\section*{Saída}

Se $X$ é o número de formas válidas de conectar as tomadas preservando as conexões já existentes, a saída deve conter um único inteiro representado o resto da divisão de $X$ por $1000000007$ ($10^9 + 7$).
 
\section*{Restrições}

\begin{itemize}
\item $1 \leq N \leq 10^6$
\item $0 \leq M \leq 10^6$
\item $1 \leq a_i \leq N$, para todo $i=1,2,\ldots,M$.
\item $1 \leq b_i \leq N$, para todo $i=1,2,\ldots,M$.
\end{itemize}
\section*{Exemplos}
\exemplo
