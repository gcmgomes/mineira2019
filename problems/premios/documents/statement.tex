% To add an image or include a .tex file you need to add
% \CWD
% to the relative (to the main document) path.
%
% Example:
% \begin{figure}
%   \centering
%   \includegraphics{\CWD/images/example.pdf}
% \end{figure}

A Maratona Mineira de Programação é um evento de muita consagração e sucesso. Há oito anos, a Maratona Mineira vem incentivando jovens estudantes a praticar programação competitiva, ajudando alunos de toda Minas Gerais a desenvolver habilidades lógicas e técnicas.

Em 2012, a primeira Maratona Mineira de Programação foi realizada em Uberlândia, e os primeiros colocados receberam prêmios muito legais! A organização deu o prêmio mais caro para o primeiro lugar, o segundo mais caro para o segundo lugar e assim por diante. Porém, dinheiro não é tudo nessa vida e Gabriel Paródia sabe disso. O time dele ficou em primeiro lugar e recebeu o prêmio mais caro. Porém, para ele, o prêmio mais legal de todos foi o dado ao $K$-ésimo time mais bem colocado.

Na Maratona de Programação temos $N$ times competindo, cada time $i$ resolveu $P_i$ problemas com uma penalidade de $T_i$ minutos. O ranking final é feito ordenando os times por ordem decrescente de problemas resolvidos, caso haja dois times com mesmo número de problemas resolvidos, o time com menor penalidade de tempo fica acima no ranking. É garantido que não temos dois times com a mesma penalidade.

Dado o número de problemas e a penalidade de cada time, qual será o time que ganhou o prêmio mais legal da competição, segundo Gabriel Paródia?


\section*{Entrada}
A primeira linha da entrada é composta por dois inteiros $N$ e $K$, representando o número de times que participaram da competição, e a posição do time que ganhou o prêmio mais legal, respectivamente. Logo após, teremos $N$ linhas. Cada linha é composta por três inteiros, $ID$, $P$ e $T$, representando o id do $i$-ésimo time, o número de problemas que o time resolveu, e a penalidade.


\section*{Saída}
A saída deverá ser composta por somente um inteiro, o id do $K$-ésimo time mais bem colocado.


\section*{Restrições}

\begin{itemize}
\item $1 \leq N \leq 1000$
\item $1 \leq K \leq N$
\item $1 \leq ID \leq N$
\item $1 \leq P_i \leq 13$
\item $1 \leq T_i \leq 2000$
\end{itemize}


\section*{Exemplos}

\exemplo
