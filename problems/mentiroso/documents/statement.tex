Todo fim de ano, Marcos vai para furnas comemorar o Natal junto com seus amigos.
Eles tem a tradição de sempre jogar \textit{Amigo Oculto} (também conhecido como 
\textit{Amigo Secreto}) nas vésperas de natal. Para quem não sabe, Amigo Oculto
é uma brincadeira na qual é sorteado para cada participante um outro participante
o qual ele deve dar um presente, de forma que cada um deles também receba um presente e que ninguém tenha que dar presente para si mesmo. 

Os amigos de Marcos inovaram e esse ano irão fazer uma surpresa para ele. Eles decidiram
que um deles irá mentir sobre quem sorteou. Marcos não participará do sorteio mas
deverá descobrir \textit{quem é o mentiroso}! Se Marcos acertar ele ficará com os
presentes de todos eles. Se errar deverá dar um presente para cada um! 

Com muito medo de errar, Marcos pediu para que fizesse um programa que dado quem
cada participante diz ter tirado, diga quem é o mentiroso. Se isso não for possível,
diga quais participantes podem estar mentindo, pois afinal, qualquer dica será
muito preciosa para Marcos!

\section*{Entrada}

A entrada contém um número inteiro $N$ que representa o número de pessoas que estão
participando do Amigo Oculto. A $i$-ésima das próximas $N$ linhas contém dois nomes separados por espaço, $a_i$ e $b_i$,
indicando que a pessoa $a_i$ sorteou a pessoa $b_i$.
Nenhum nome é muito esquisito, tendo no máximo 10 letras.

\section*{Saída}

A primeira linha da saída deve conter um inteiro $M$ representando o número de possíveis mentirosos. Cada uma das próximas $M$ linhas deve
conter o nome de um deles. Apresente os nomes em ordem alfabética.

\section*{Restrições}

\begin{itemize}
\item $2 \le N \le 10^5$.
\item $1 \le |a_i|, |b_i| \le 10$, para todo $i=1,2,\ldots,N$.
\item Nomes contém apenas letras minúsculas. 
\item Os nomes dos $N$ participantes são distintos.
\end{itemize}

\section*{Exemplos}

\exemplo
