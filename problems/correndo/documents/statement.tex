José gosta de correr em praças circulares da cidade. Porém, ele prefere correr em linha reta. Além disso, ele prefere mudar de direção o mínimo de vezes possível. As praças são formadas por um limite circular exterior (a divisão entre a praça e a rua) e um limite circular interior (que cerca um espaço aberto no centro da praça). Se ele vai correr na área entre esses dois círculos, qual é o mínimo de vezes que ele precisa mudar de direção para, correndo sempre em linha reta, sair de algum ponto, contornar o espaço central da praça e voltar ao ponto de partida, sem sair da praça? Entrada: dois inteiros a e b.

\section*{Entrada}

A entrada contém apenas uma linha com dois inteiros $a$ e $b$, separados por espaço, representando os raios das circunferências que formam a praça.

\section*{Saída}

Escreva na saída uma linha contendo um único inteiro $N$, o menor número possível de vezes que José deverá trocar de direção em sua corrida.

\section*{Restrições}

$$? \leq a < b \leq ?$$

\section*{Exemplos}
\exemplo
