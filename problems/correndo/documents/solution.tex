É fácil ver que a solução ótima pode sempre caminhar entre pontos que estão na extremidade do círculo maior
(dá para transformar qualquer solução numa desse tipo sem aumentar o número de lados). Além disso, o ótimo é
fazer com que cada troca de direção tenha o maior ângulo possível. Basta calcular esse ângulo e pegar o teto
da divisão de 360 por ele, subtraindo um (o problema pede o número de trocas de direção durante a corrida).
Para calcular o ângulo, basta desenhar no papel as duas circunferências, centradas no ponto O, e de raios
$a$ e $b$. Sejam $P_1$ e $P_2$ dois pontos sobre a circunferência de raio $b$, de tal forma que o segmento
$P1 - P2$ tangencie a circunferência de raio $a$ (assim o arco $P1 - P2$ é o maior possível) num ponto que
chamamos de $S$. Observe o triângulo $OSP1$. Ele é retângulo, e um cateto tem tamanho $a$ e a hipotenusa tem
tamanho $b$. O seno do ângulo $OP1S$ é $a / b$. Com isso, calculamos o ângulo com a função arco seno. Com isso,
é fácil chegar à resposta. Calculamos o outro ângulo agudo do triângulo, e multiplicando por 2 temos o tamanho do
arco $P1 - P2$. Basta agora pegar o teto da divisão de $2*\pi$ por esse arco para termos o número de lados do
polígono que José irá percorrer. Subtraindo 1, temos a resposta do problema.
