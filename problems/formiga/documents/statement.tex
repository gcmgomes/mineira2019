Júlia gosta muito de animais domésticos. Ela tem vários cachorros, gatos, papagaios, mas seu animal preferido mesmo é a formiga. Júlia está construindo um brinquedo para sua formiga. Ela começou pegando dois pedaços de madeira de 1 metro cada uma e construindo um degrau, como mostra a figura da esquerda nesta imagem: https://www.dropbox.com/s/vfjw2l4nn9y2qag/Design.png?dl=0

A formiga de Júlia sempre começa do ponto mais baixo do degrau, e sobe os segmentos do degrau até chegar ao topo. Ela não pára por aí: ao chegar ao topo, ela começa a descer, depois sobe novamente, e assim segue, até ter fome. Neste momento, a formiga fica sem energia e precisa se alimentar. Júlia, como boa dona, pega então alguns grãos de açúcar e leva até a boca de seu animal de estimação.

Para que o brinquedo fique menos monótono, Júlia decidiu transformar cada degrau em dois, conforme mostra a figura. Ela repetiu esse procedimento k vezes. Cada vez que Júlia faz isso, o número de degraus do brinquedo dobra, e o tamanho de todos os segmentos fica exatamente metade do tamanho dos segmentos anteriormente. Na figura 2 (ainda não feita), você pode ver como o brinquedo ficou depois de 2 e 3 repetições desse procedimento.

Júlia notou que sua formiga demora 2 segundos para subir um segmento, 1 segundo para percorrer um segmento horizontal, e apenas meio segundo para descer um segmento, independente do tamanho dele. A formiga já está brincando há n segundos, e é hora de dar comida para ela. Você pode dizer exatamente onde está a formiga agora?

\section*{Entrada}

A entrada contém apenas uma linha com os dois inteiros $K$ e $N$, separados por espaço, representando o número de transformações sofridas pelo brinquedo e o número de segundos que Júlia deixou a formiga brincar.

\section*{Saída}

Saída: x y (x é a posição x da formiga, como fração irredutível se for necessário, e y é a posição y)

\section*{Restrições}

$$0 \leq K \leq 40$$
$$1 \leq N \leq 10^{18}$$

\section*{Exemplos}
\exemplo
