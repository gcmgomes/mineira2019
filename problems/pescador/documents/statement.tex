História de Pescador.

Senhor Figueiredo é um pescador famoso nos botecos da cidade de Lavras por causa das histórias duvidosas que ele conta sobre as aventuras dele no Rio Grande. Certa vez o jovem Ceconelli, vizinho do senhor Figueiredo, resolveu investigar a veracidade dos causos. De forma insistente, Ceconelli pediu ao senhor Figueiredo que o acompanhasse nas aventuras pesqueiras. Para evitar a fadiga semanal, o senhor Figueiredo deixou que o jovem o acompanhasse, desde que carregasse os mantimentos e a vara dele. Durante a pesca, Ceconelli acabou descobrindo o segredo do senhor Figueiredo sobre as histórias de pescador em relação a quantidade de peixes pescados. Ele percebeu que nas histórias, o senhor Figueiredo multiplica a quantia fisgada por um fator fixo. Visando ficar também popular na cidade, Ceconelli quer contribuir com as mirabolantes histórias contadas nos butecos pelo senhor Figueiredo. Para isso, ele solicitou que se criasse um programa que dado um valor inteiro fixo e a quantidade de peixes pescados de cada espécie, seja informado a quantia mentirosa de pescados.

\section*{Entrada}

A primeira linha da entrada é composta pelos inteiros $M$ e $N$, que representam, respectivamente, a quantidade de espécies pescadas e o fator usado pelo senhor Figueiredo nas histórias. Na segunda linha há $M$ inteiros $Q_i$, $i \in \{1, \ldots, M\}$, separados por espaço, que representam a quantidade real de peixes pescados por cada espécie.

\section*{Saída}

A saída apresenta em cada linha o valor mentiroso para cada espécie no formato "Peixe $I$: $V$", em que $I$ representa o índice da espécie e $V$ a quantidade mentirosa.

\section*{Restrições}

$$1 \leq M \leq 100$$
$$2 \leq N \leq 1000$$
$$1 \leq Q_i \leq 100$$


\section*{Exemplos}

Entrada

3 10
2 10 5

Saída

Peixe 1: 20
Peixe 2: 100
Peixe 3: 50


Entrada

4 100
1 2 3 4

Saída

Peixe 1: 100
Peixe 2: 200
Peixe 3: 300
Peixe 4: 300


Entrada

2 2
2 2

Saída

Peixe 1: 4
Peixe 2: 4

\exemplo
