Senhor Figueiredo é um pescador famoso nos botecos da cidade de Lavras por causa das histórias duvidosas que ele conta sobre as aventuras dele no Rio Grande.
Certa vez o jovem Ceconelli, vizinho do senhor Figueiredo, resolveu investigar a veracidade dos causos.
Depois de muita insistência de Ceconelli, o senhor Figueiredo finalmente deixou que o jovem o acompanhasse, desde que carregasse os mantimentos e sua vara de pesca.
Durante a pesca, Ceconelli acabou descobrindo o segredo do senhor Figueiredo sobre as histórias do pescador; o senhor Figueiredo multiplica a quantia fisgada por um fator fixo!
Dados o fator multiplicativo usado pelo senhor Figueiredo e a quantidade real de peixes pescados de cada espécie, qual será a quantidade de peixes, de cada espécie, que o velho pescador irá contar para seus amigos no bar?

\section*{Entrada}

A primeira linha da entrada é composta pelos inteiros $M$ e $N$, que representam, respectivamente, a quantidade de espécies pescadas e o fator usado pelo senhor Figueiredo nas histórias.
As próximas $M$ linhas contém um inteiro cada uma.
Na $i$-ésima linha encontra-se o inteiro $Q_i$, representando a quantidade real de peixes pescados da espécie $i$.

\section*{Saída}

A saída deve conter $M$ linhas.
Na $i$-ésima linha você deve imprimir a frase "Peixe $i$: $V_i$", sem aspas, onde $V_i$ é a quantidade de peixes da espécie $i$ que o senhor Figueiredo diz ter pescado.

\section*{Restrições}

$$1 \leq M \leq 100$$
$$2 \leq N \leq 1000$$
$$1 \leq Q_i \leq 100$$


\section*{Exemplos}

\exemplo
