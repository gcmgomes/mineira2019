Há vários anos, Bacon -- o Bravo -- saiu vitorioso das disputas de sucessão do reino das capivaras.
Infelizmente, Bacon já está com 16 anos, uma idade avançada para uma capivara.
Como todo bom rei absolutista, Bacon -- o Bravo -- também gostaria de ser conhecido como fundador
da maior dinastia já vista por sua espécie, e almeja colocar no trono sua filha mais velha,
Bacon -- a Ambiciosa.

Para evitar um novo conflito entre seus vassalos, o monarca convocou todos os nobres de seu reino
para um jantar onde será decidido o futuro moncarca das capivaras, onde os convidados serão dispostos em uma
grande mesa retangular com a família real ocupando as duas cabeceiras.
Para deixar os nobres mais confortáveis, e assim aumentar suas chances de realizar seus sonhos,
Bacon -- o Bravo -- quer dividir as nobre capivaras nas duas laterais da mesa tal que cada capivara tenha,
no máximo, um de seus aliádos políticos do outro lado.
Infelizmente, ou felizmente para nosso monarca, as richas entre os nobres ainda estão muito fortes e evidentes,
e com isso cada nobre tem não mais do que três aliados.
Os lados da mesa não precisam ter o mesmo número de ocupantes, mas cada um deve ter pelo menos uma capivara.

Ajude o rei a determinar se é possível ou não dividir os nobres dessa forma e seja imortalizado nas 
\textit{Crônicas das Capivaras} como Humana/Humano -- a/o Ajudante!

\section*{Entrada}

A primeira linha da entrada contém dois inteiros separados por espaços, $N$ e $M$, o número de nobres e o número
de aliaças políticas existentes entre eles.
Seguem $M$ linhas, cada uma com dois inteiros $u_i$ e $v_i$, indicando que a capivara $u_i$ aliada à capivara $v_i$.
Estamos falando de um reino, então não existem capivaras isoladas politicamente: sempre é possível chegar de uma capivara
a outra apenas seguindo os elos de lealdade.

\section*{Saída}

Na primeira linha da saída escreva um único inteiro $R$, que deve ser $0$ caso seja impossível realizar a divisão,
ou $1$, casa seja possível.
Se $R = 0$, não há mais linhas na entrada.
Se $R = 1$, a segunda linha começa com um único inteiro $S_1$, o número de capivaras do lado esquerdo da mesa.
Seguem então $S_1$ inteiros nessa linha, separados por espaço, cada um representando o número de uma capivara que está do lado esquerdo.
A terceira linha começa com um único inteiro $S_2$, o número de capivaras do lado direito da mesa.
Seguem então $S_2$ inteiros nessa linha, separados por espaço, cada um representando o número de uma capivara que está do lado direito.


\section*{Restrições}

$$2 \leq N \leq 10^{6}$$
$$1 \leq M \leq \frac{3N}{2}$$
$$1 \leq u_i, v_i \leq N$$

\section*{Exemplos}
\exemplo
