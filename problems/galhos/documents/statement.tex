O natal já passou. Porém, por causa da correria do dia-a-dia, a árvore de natal na casa de Graça continua montada.
Mas a família dessa garota-prodígio decidiu fazer um mutirão no próximo fim de semana para desmontar a árvore,
onde também ainda está a toca do coelhinho da páscoa.

Como Ada, a mãe de Graça, é uma cientista da computação apaixonada pelo que faz,
ela nunca perde uma oportunidade de introduzir a filha
no fabuloso mundo dos algoritmos e estruturas de dados.
Por isso, a árvore de natal da familia foi realmente montada como uma árvore
que você encontraria em um curso de algoritmos: ela possui bolas numeradas de $1$ a $N$, cada uma
podendo ser vista como um vértice, pendurada a outra bola acima dela,
com exceção da raiz que foi pendurada no teto.

Para que o desmonte da árvore seja mais divertido, Ada propôs um desafio a Graça.
No $i$-ésimo segundo do desmonte, Ada irá cortar o fio que originalmente mantinha a bola de número $B_i$ suspensa.
Caso a bola ainda esteja suspensa, ela irá então cair ao chão, juntamente com todas as bolas que estavam
penduradas nela direta ou indiretamente.
Pode acontecer que Ada corte o fio de uma bola que já esteja no chão - neste caso, nada acontece.
Passados $N$ segundos, com todas as bolas no chão, Ada pergunta a Graça em qual segundo cada uma das $N$
bolas caiu.
Caso Graça acerte todos os números, ela será recompensada com uma viagem para assistir a próxima
edição da Maratona Mineira, onde ela poderá conhecer outras jovens mentes promissoras como a sua!

Graça é muito ágil e conseguiu anotar todos os números necessários manualmente. Mas será que você
consegue fazer um programa para ajudá-la a se preparar para o desmonte de árvore de natal do ano que vem?
Considere que as bolas caem imediatamente ao chão quando deixam de estar presas direta ou indiretamente ao teto.

\section*{Entrada}

A primeira linha da entrada contém um inteiro $N$, o número de bolas da árvore de natal.
A segunda linha contém $N$ inteiros $P_1, \cdots, P_N$ separados por espaço.
$P_i$ indica que há um fio prendendo a bola $i$ à bola $P_i$, que está acima na árvore.
$P_i = -1$ indica que a bola $i$ é a raiz da árvore.
Há sempre exatamente uma bola $i$ com 
$P_i = -1$, e é garantido que as ligações dadas descrevem uma árvore.
A última linha contém $N$ inteiros $B_1, \cdots, B_N$ separados por espaço, indicando qual fio é
cortado no $i$-ésimo segundo do processo de desmonte.
Cada inteiro entre $1$ e $N$ aparece exatamente uma vez na última linha da entrada.

\section*{Saída}

Escreva na saída uma linha com $N$ inteiros separados por espaço.
O $i$-ésimo desses inteiros deve indicar em qual segundo a bola $i$ cai ao chão.

\section*{Restrições}

$$1 \leq N \leq 10^{6}$$

\section*{Exemplos}
\exemplo
