É dada uma árvore, e uma sequência de $N- 1$ vértices que se desprendem do pai e caem da árvore. O $i$-ésimo vértice da sequência cai da árvore no momento $i$, juntamente com toda a sua subárvore (fora os vértices que já caíram antes). Para todos os vértices, em que momento eles caem?

É dada o número $N$: o tamanho da árvore. Seguem n-1 inteiros. O i-ésimo inteiro é o pai do vértice de número i + 1. O vértice 0 é a raiz da árvore. Seguem, então, n inteiros, com a ordem em que os vértices se desprendem do seu pai na árvore. A saída deve conter n inteiros. Para cada vértice i, deve-se imprimir em que momento o vértice i cai da árvore.

\section*{Entrada}

A primeira linha da entrada contém dois inteiros $N$, o número de vértices da árvore.
A segunda linha contém $N-1$ inteiros separados por espaço, indicando os momentos em que cada um do vértices

\section*{Saída}

Escreva na saída $N-1$ inteiros. O $i$-ésimo desses inteiros deve indicar em qual momento o vértice $i$ cai da árvore.
\section*{Restrições}

$$1 \leq N \leq 10^{6}$$

\section*{Exemplos}
\exemplo
