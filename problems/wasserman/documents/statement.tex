Em sua última visita a Furnas, Wasserman o Rei das Águas, ficou bastante confuso com o 
dialeto mineirês que os peixes falavam. Bagri, o peixe poliglota, foi convocado para ajudar 
Wasserman a entender esse novo idioma. Bagri passou muitos anos estudando o dialeto, e 
conhecia todas as palavras e frases mais comuns do Peixe-mineirês de Furnas. Fez então uma 
lista com todas elas e a deu à Wasserman. Aqui podemos ver um trecho desta lista:

\begin{align*}
blohblo:& \text{ aos competidores}\\
bloh:& \text{ ajuda}\\
blobla:& \text{ aos artistas da}\\
ubl:& \text{ Peça}\\
blaubl:& \text{ da Maratona Mineira}\\
\end{align*}

Então Bagri advertiu:
- Sempre espere um peixe terminar sua frase, pois nem sempre uma palavra é o que parece!
No peixe-mineirês quando há mais do que uma possível tradução, o correto é aquela em que 
as primeiras palavras são as maiores possíveis!

Wasserman, confuso, perguntou:
- Quem poderá me ajudar a traduzir tal idioma?

Bagri, em peixe-mineirês, respondeu:
- ublblohbloblaubl

Wasserman pensou, pensou, pensou até que disse:
- A tradução para isso é \textit{Peça ajuda aos artistas da Peça} ? Que Peça?

Bagri lamentando-se, disse:
- Você está ficando velho Wasserman, não está entendendo muita coisa. Você traduziu como:

\begin{align*}
ubl + bloh + blobla + ubl: \text{Peça} + \text{ajuda} + \text{aos artistas da} + \text{Peça}
\end{align*}

- Porém o correto é:

\begin{align*}
ubl + blohblo + blaubl: \text{Peça} + \text{aos competidores} + \text{da Maratona Mineira}
\end{align*}

- Pois em ambas traduções a primeira palavra tem o mesmo tamanho, porém na segunda tradução, a segunda
palavra \textit{blohblo} é maior do que a segunda palavra da primeira tradução \textit{bloh}. Logo o
correto é \textit{Peça aos competidores da Maratona Mineira}!!!

Wasserman então pediu para que você faça um programa que traduza o peixe-mineirês.

\section*{Entrada}

A primeira linha da entrada contém o inteiro $N$, representando o número de palavras de peixe-mineirês consideradas.
As $2N$ linhas seguintes contém as palavras e suas traduções. A $(2i - 1)$-ésima dessas linhas contém a $i$-ésima palavra em peixe-mineirês $n_i$, enquanto a $2i$-ésima delas contém a tradução $t_i$ da palavra $n_i$. Uma tradução é uma sequência de uma ou mais palavras separadas por espaço. Em seguida, a entrada contém mais uma linha apenas com letras (sem espaços) representando a sentença $S$ a ser traduzida.

\section*{Saída}

A saída deve conter uma única linha com a tradução de $S$. A tradução deve ser uma sequência de palavras separadas por espaço.

\section*{Restrições}

\begin{itemize}
\item $1 \le N \le 5 \times 10^4$.
\item $1 \le |n_i| \le 50$ para todo $i = 1, 2, \ldots, n$.
\item $1 \le |t_i| \le 100$ para todo $i = 1, 2, \ldots, n$.
\item $1 \le |S| \le 10^6$.
\item Todas as palavras $n_i$ são distintas.
\item $S$ contem apenas letras e possuí uma tradução válida.
\item As palavras na entrada são formadas apenas por letras a-z ou A-Z.
\end{itemize}

\exemplo
