Em sua última visita a Furnas, Wasserman o Rei das Águas, ficou bastante confuso com o 
dialeto mineirês que os peixes falavam. Bagri, o peixe poliglota, foi convocado para ajudar 
Wasserman a entender esse novo idioma. Bagri passou muitos anos estudando o dialeto, e 
conhecia todas as palavras e frases mais comuns do Peixe-mineirês de Furnas. Fez então uma 
lista com todas elas e a deu à Wasserman. Aqui podemos ver um trecho desta lista:

\begin{align*}
blohblo:& \text{ aos competidores}\\
bloh:& \text{ ajuda}\\
blobla:& \text{ aos artistas da}\\
ubl:& \text{ Peça}\\
blaubl:& \text{ da Maratona Mineira}\\
\end{align*}

Então Bagri advertiu:
- Sempre espere um peixe terminar sua frase, pois nem sempre uma palavra é o que parece!
No peixe-mineirês quando há mais do que uma possível tradução, o correto é aquela em que 
as primeiras palavras são as maiores possíveis!

Wasserman, confuso, perguntou:
- Quem poderá me ajudar a traduzir tal idioma?

Bagri, em peixe-mineirês, respondeu:
- ublblohbloblaubl

Wasserman pensou, pensou, pensou até que disse:
- A tradução para isso é \textit{Peça ajuda aos artistas da Peça} ? Que Peça?

Bagri lamentando-se, disse:
- Você está ficando velho Wasserman, não está entendendo muita coisa. Você traduziu como:

\begin{align*}
ubl + bloh + blobla + ubl: \text{Peça} + \text{ajuda} + \text{aos artistas da} + \text{Peça}
\end{align*}

- Porém o correto é:

\begin{align*}
ubl + blohblo + blaubl: \text{Peça} + \text{aos competidores} + \text{da Maratona Mineira}
\end{align*}

- Pois em ambas traduções a primeira palavra tem o mesmo tamanho, porém na segunda tradução, a segunda
palavra \textit{blohblo} é maior do que a segunda palavra da primeira tradução \textit{bloh}. Logo o
correto é \textit{Peça aos competidores da Maratona Mineira}!!!

Wasserman então pediu para que você faça um programa que traduza o peixe-mineirês.

\section*{Entrada}

A entrada contém um número inteiro $N$, seguido de $2N$ linhas com traduções de palavras
do peixe-mineirês, no seguinte formato: a linha $i$ contém uma palavra $n_i$ de peixe-mineirês 
e a linha abaixo ($i+1$) contém uma ou mais palavras $t_i$ correspondentes 
a tradução.
Por último, na linha $2N + 2$, uma 
sentença $S$ com apenas letras minúsculas do peixe-mineirês é dada. 

\section*{Saída}

Escreva a tradução, acrescentando um espaço estre as palavras traduzidas.

\section*{Restrições}

$1 \le N \le 50000$

$1 \le |n_i| \le 50$

$1 \le |t_i| \le 100$

$1 \le |S| \le 1000000$ 

É garantido que toda sentença $S$ tem tradução. É também garantido
que toda palavra de peixe-mineirês tem apenas uma tradução.
Toda string na entrada contém apenas caracteres da tabela ASCII.

\exemplo
