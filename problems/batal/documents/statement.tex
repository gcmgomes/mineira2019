Bibika está jogando o famoso jogo Navalha Batal. Para quem não o conhece, o jogo é basicamente jogado entre duas pessoas onde cada uma possui um tabuleiro de tamanho NxN com algumas peças (que representam Navios) de tamanhos 1xT e Tx1 previamente inseridas, onde o valor de T pode ser qualquer número natural positivo. Vale ressaltar que não se pode sobrepor peças e toda peça deve estar totalmente inserida no tabuleiro, não podendo ter parte(s) para fora. 

Após posicionarem suas peças iniciais, cada jogador tem alguns minutos para analisar o tabuleiro do adversário e devem calcular (ou chutar) a quantidade de peças 1xT e Tx1 que ainda podem ser inseridas. Ganha quem chegar mais perto da resposta.

// Coloco uma imagem de um tabuleiro pequeno e mostro quantas possibilidades de peças existem pra serem colocadas, para deixar totalmente claro 

Após jogar com tabuleiro de tamanho consideravelmente pequeno, Bibika gostaria de saber as respostas para tabuleiros maiores. Como é impossível a olho nu, ela dará as posições dos navios inicias e cabe a você ajudá-la!

\section*{Entrada}

A primeira linha contém dois inteiros $N$ e $Q$, sendo $NxN$ o tamanho do tabuleiro e $Q$ a quantidade de células que estão previamente preenchidas com navios.
As próximas $Q$ linhas possuem dois inteiros, $X_i$ e $Y_i$, indicando que a coordenada $X_i, Y_i$ do tabuleiro está preenchida.

\section*{Saída}

Exiba um único inteiro, a quantidade de navios de tamanho $1xT$ ou $Tx1$ que são possíveis de serem colocados, dado que fiquem totalmente inseridas no tabuleiros e não exista sobreposição com outros navios. Como a quantidade pode ser muito grande, exiba módulo $10^9+7$.

\section*{Restrições}

$1 \leq N \leq 10^6$
$0 \leq Q \leq 2x10^5$
$1 \leq X, Y \leq N$

\section*{Exemplos}
\exemplo
